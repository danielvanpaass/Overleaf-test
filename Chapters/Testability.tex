\section{Testability}
\subsection{Controllability and Observability}
Controllability defines the difficulty or cost of setting a certain output or node to a value of choice (be it 1 or 0)\cite{CandO}. This controllability decreases when the system that is being tested becomes more complex. A single node has the highest controllability of 1. This is because the input and output are directly connected to each other and thus are the same. \\
Another value which is useful for testability is observability. Observability measures the cost of observing any value on a node\cite{CandO}. The observability, just like the controllability, decreases as the system that is being tested increases in complexity. Although there is a correlation between observability and controllability, they are not the same or inverses of each other.

\subsection{Black-box and white-box testing}
Black-box and white-box testing are two very important forms of testing which are still useful to know the difference of. Black-box testing is testing without looking at what is happening between components or parts of the design. This way of testing is a good way if it is thought that the system is functional in the way it is intended to be. \cite{informationvisualization}. If this is not the case, it may become evident what error is by looking at the type of output error the system is producing. If this also fails,  it is advisable to resort to white-box tests.\\
White-box or clear-box testing is testing parts of your design individually. This helps to easily see where the error in the system is. By separating parts of the system into subsystems,  it is advisable to resort to white-box tests and thus increase the controllability and observability. 
Black-box and white-box testing are meant to be used together\cite{informationvisualization}. Black-box testing is easier and less time consuming than white-box testing, but white-box testing goes in much more depth than black-box testing and makes spotting the error easier. In the beginning of this project white-box testing will mostly be used since most subsystems still need to be built. When everybody has made their subsystem and all subsystems are combined to make the large system, black-box testing is possible and be used to check for errors often.